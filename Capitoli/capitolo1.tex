\section{Introduzione generale al progetto}
Come ogni anno, alla scuola cantonale di commercio, gli allievi devono svolgere un lavoro di maturità a gruppi, chiamato anche comunemente progetto interdisciplinare. Questo lavoro viene svolto attraverso un processo di ricerca che comprende la raccolta di materiale, analisi di dati statistici, interviste, sondaggi, ricerche approfondite in rete, lettura di testi e documenti che possano rispondere ad una domanda di ricerca precisa, che è stata fissata all’inizio del corso.
L’argomento principale, sul quale è stato svolto il nostro lavoro, è “l’Economia a km 0”. Per introdurre il corso abbiamo visionato un film documentario dal titolo “Domani”, il quale ci ha permesso di identificare diverse possibili tematiche come la riqualificazione urbana, gli orti urbani, la moneta locale e gli scambi dal produttore al consumatore, tra quest’ultime abbiamo focalizzato la nostra attenzione sulla riqualificazione urbana del nuovo agglomerato di Bellinzona.
Ci siamo interrogati su quali possono essere le esigenze della popolazione e quali sono le migliorie che possono essere apportate al nuovo agglomerato per il bene comune; dunque abbiamo voluto approfondire le nostre conoscenze sull’urbanistica e su come riqualificare il territorio. 

