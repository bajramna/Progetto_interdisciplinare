\section{La riqualificazione urbana}
\subsection{Definizione}
Prima di addentrarsi nel vivo dell’argomento è necessario dare una definizione precisa di che cosa sia la riqualificazione urbana e in seguito che cosa significhi per noi.
Con riqualificazione urbana si intende, in primo luogo, dare una nuova immagine allo spazio edilizio già esistente in una determinata zona geografica. Si possono racchiudere in essa diversi interventi che fanno parte di una rigenerazione, ad esempio nelle zone più degradate, interventi che possano limitare il consumo di territorio ma allo stesso tempo che rispettino l’ambiente e il paesaggio circostante, soprattutto nell’ambito della sostenibilità. Le funzioni urbanistiche sono molto importanti e comprendono diversi aspetti come l’organizzazione delle parti urbane e il collegamento con le zone periferiche, la creazione di nuove centralità di svago, la revisione delle infrastrutture stradali e pedonali.
Spesso la condizione delle moderne periferie non è delle migliori, diversi sono i motivi che portano ad avere una situazione simile, basti pensare alla disomogenea organizzazione morfologica, il collegamento assente fra periferia e città, ma soprattutto una gestione poco efficacie degli spazi. 

\subsubsection{L'urbanistica}

\subsubsection{Mobilità e sicurezza}

\subsection{Cosa significa per noi}
\subsubsection{Definizione di benessere}

\subsection{Casi ed esempi reali}
\subsubsection{riqualifica della foce di Cassarate}
\subsubsection{riqualifica delle Lorelei }